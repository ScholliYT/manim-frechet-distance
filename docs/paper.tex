\documentclass[
oneside,
fontsize=11pt
]{scrartcl}



%%%%%%%%%%%%%%%%%%%%%%%%%%%%%%%%%%%%%%%%%%%%%%%%%%%%%%%%%%%%%%%%%%%%%%%%%%%%%%%%
%%%
%%% packages
%%%

%%%
%%% encoding and language set
%%%


\usepackage[english]{babel}
\usepackage[a4paper]{geometry}

%%% fontenc, ae, aecompl: coding of characters in PDF documents
\usepackage[utf8]{inputenc}
\usepackage[T1]{fontenc}
\usepackage{lmodern}

\usepackage[autostyle=true,german=quotes]{csquotes}
\usepackage{caption}
\usepackage{subcaption}
\usepackage{minted}


%%%
%%% technical packages
%%%

%%% amsmath, amssymb, amstext: support for mathematics
\usepackage{amsmath,amssymb,amstext}





%%%
%%% Colors
%%%

\RequirePackage[table,dvipsnames]{xcolor}


\definecolor{TUGreen}{rgb}{0.517,0.721,0.094}


%%%
%%% PDF
%%%

\usepackage{ifpdf}

%%% Should be LAST usepackage-call!
%%% For docu on that, see reference on package ``hyperref''
\ifpdf%   (definitions for using pdflatex instead of latex)

  %%% graphicx: support for graphics
  \usepackage[pdftex]{graphicx}

  \pdfcompresslevel=9

  %%% hyperref (hyperlinks in PDF): for more options or more detailed
  %%%          explanations, see the documentation of the hyperref-package
  \usepackage[%
    %%% general options
    pdftex=true,      %% sets up hyperref for use with the pdftex program
    %plainpages=false, %% set it to false, if pdflatex complains: ``destination with same identifier already exists''
    %
    %%% extension options
    backref,      %% adds a backlink text to the end of each item in the bibliography
    pagebackref=false, %% if true, creates backward references as a list of page numbers in the bibliography
    colorlinks=true,   %% turn on colored links (true is better for on-screen reading, false is better for printout versions)
    linkcolor=TUGreen,
    urlcolor=TUGreen,
    %
    %%% PDF-specific display options
    bookmarks=true,          %% if true, generate PDF bookmarks (requires two passes of pdflatex)
    bookmarksopen=true,     %% if true, show all PDF bookmarks expanded
    bookmarksnumbered=true, %% if true, add the section numbers to the bookmarks
    %pdfstartpage={1},        %% determines, on which page the PDF file is opened
    pdfpagemode=None         %% None, UseOutlines (=show bookmarks), UseThumbs (show thumbnails), FullScreen
  ]{hyperref}


  %%% provide all graphics (also) in this format, so you don't have
  %%% to add the file extensions to the \includegraphics-command
  %%% and/or you don't have to distinguish between generating
  %%% dvi/ps (through latex) and pdf (through pdflatex)
  \DeclareGraphicsExtensions{.pdf}

\else %else   (definitions for using latex instead of pdflatex)

  \usepackage[dvips]{graphicx}

  \DeclareGraphicsExtensions{.eps}

  \usepackage[%
    dvips,           %% sets up hyperref for use with the dvips driver
    colorlinks=false %% better for printout version; almost every hyperref-extension is eliminated by using dvips
  ]{hyperref}

\fi


%%% sets the PDF-Information options
%%% (see fields in Acrobat Reader: ``File -> Document properties -> Summary'')
%%% Note: this method is better than as options of the hyperref-package (options are expanded correctly)
\hypersetup{
  pdftitle={Fréchet Distance}, %%
  pdfauthor={Tom Stein}, %%
  pdfsubject={Seminar Algorithm Engineering 22/23}, %%
  pdfcreator={Accomplished with LaTeX2e and pdfLaTeX with hyperref-package.}, %% 
  pdfproducer={}, %%
  pdfkeywords={} %%
}



%%%%%%%%%%%%%%%%%%%%%%%%%%%%%%%%%%%%%%%%%%%%%%%%%%%%%%%%%%%%%%%%%%%%%%%%%%%%%%%%
%%%
%%% define the titlepage
%%%

% \subject{}   %% subject which appears above titlehead
% \titlehead{} %% special heading for the titlepage

%%% title
\title{Fréchet Distance}

%%% author(s)
\author{Tom Stein}

%%% date
\date{November 27, 2022}


%%%%%%%%%%%%%%%%%%%%%%%%%%%%%%%%%%%%%%%%%%%%%%%%%%%%%%%%%%%%%%%%%%%%%%%%%%%%%%%%
%%%
%%% begin document
%%%

\begin{document}

% \pagenumbering{roman} %% small roman page numbers

%%% include the title
% \thispagestyle{empty}  %% no header/footer (only) on this page
%  \maketitle

% Titlepage ---------------------------------------------------------
%
\pdfbookmark{Titelpage}{pdf:title}
\newgeometry{
    a4paper,
    top=25mm,
    bottom=25mm,
    left=20mm,
    right=20mm,
}
\begin{titlepage}
    \includegraphics[width=0.4\textwidth]{images/tud_logo_rgb.jpg}

    \begin{center}
        \vspace{3.5cm} \LARGE Seminar Algorithm Engineering 22/23

        \vspace{0.5cm} \huge \textbf{Fréchet Distance}

        \vspace{5cm} \textbf{Name} % \textbf{Tom Stein}

        \vspace{0.25cm} \Large December 05, 2022
    \end{center}

    \vspace{5.2cm} \large \noindent Supervisor: \\
    Dr. Carolin Rehs
    
    \vspace{1cm} \noindent Technische Universität Dortmund \\
    Department of Computer Science \\
    Chair 11 (Algorithm Engineering) \\ 
    \url{https://ls11-www.cs.tu-dortmund.de/}

    
\end{titlepage}



%%% start a new page and display the table of contents
% \newpage
% \tableofcontents

%%% start a new page and display the list of figures
% \newpage
% \listoffigures

%%% start a new page and display the list of tables
% \newpage
% \listoftables

%%% display the main document on a new page 
\newpage

% \pagenumbering{arabic} %% normal page numbers (include it, if roman was used above)

%%%%%%%%%%%%%%%%%%%%%%%%%%%%%%%%%%%%%%%%%%%%%%%%%%%%%%%%%%%%%%%%%%%%%%%%%%%%%%%%
%%%
%%% begin main document
%%% structure: \section \subsection \subsubsection \paragraph \subparagraph
%%%

\newgeometry{left=3cm, right=4.5cm, top=4cm, bottom=4cm}

\section*{Abstract}


\section{Introduction}
% motivation

% two (polygonal) curves

% Hausdorff distance
The commonly known \textit{Hausdorff distance} can be understood 
as the largest distance between any two points on the curves 
and is calculated by \autoref{eq_hausdorff_distance} \cite{alt_computing_1995}.

\begin{align}
  \label{eq_hausdorff_distance}
  \delta_{hd}(P,Q) = \max \left( \sup_{p \in P} \inf_{q \in Q} \text{dist}(p,q), \sup_{q \in Q} \inf_{p \in P} \text{dist}(p,q) \right)
\end{align}

However, this metric does not take the course of the curves into account. 
Two curves may be very dissimilar in the preception of a human 
but still have a small Hausdorff-distance, see \autoref{fig_hausdorff_distance_bad_example}.

\begin{figure}[ht]
  % TODO: insert figure
  \caption[Hausdorff distance of two dissimilar curves]{The two shown curves are dissimilar (in a common sense) depsite having a small Hausdorff distance.}
  \label{fig_hausdorff_distance_bad_example}
\end{figure}


% Fréchet distance motivation

% continuous Fréchet distance equation

% Scope



% Fréchet distance Variants 
\section{Variants}
Through the years many slightly differnt variants of the continuous Fréchet distance have evolved. 
The continuous Fréchet distance takes into account every single point on the curves, 
with interpolation in the case of polygonal curves, 
without allowing to walk backwards 
and without any obstacles in between the two curves.

% TODO: Why allow it? Better algorithms?
The \textit{weak Fréchet distance} slightly modifies the continuous version by allowing 
going backwards. 

% TODO: Is that mapping bijective? Add sources
The \textit{discrete Fréchet distance} 
This gives an upper bound on the continuous Fréchet distance % Source?
and is often used as a approximation. % Source?
% discrete Fréchet distance and contrast to hausdorff distance

% (Geodesic Fréchet distance)

% (Homotopic Fréchet distance)



\section{Algorithm}
% Original algorithm for the continuous Fréchet distance

% Free Space Diagram


% Runtime analysis (?)
% Correctness proof (?)


\section{Conclusion}





%%%
%%% end main document
%%%
%%%%%%%%%%%%%%%%%%%%%%%%%%%%%%%%%%%%%%%%%%%%%%%%%%%%%%%%%%%%%%%%%%%%%%%%%%%%%%%%

\newpage
\appendix  %% include it, if something (bibliography, index, ...) follows below

%%%%%%%%%%%%%%%%%%%%%%%%%%%%%%%%%%%%%%%%%%%%%%%%%%%%%%%%%%%%%%%%%%%%%%%%%%%%%%%%
%%%
%%% bibliography
%%%
%%% available styles: abbrv, acm, alpha, apalike, ieeetr, plain, siam, unsrt
%%%
\bibliographystyle{alpha}

%%% name of the bibliography file without .bib
%%% e.g.: literature.bib -> \bibliography{literature}
\bibliography{literature}

\end{document}
%%% }}}
%%% END OF FILE
